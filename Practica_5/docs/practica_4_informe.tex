\input{Preamblev2.sty}
\usepackage{multirow}
\begin{document}

\title{Práctica 5: Keras}
\author{Evelyn G. Coronel}

\affiliation{Redes Neuronales y Aprendizaje Profundo para Visión Artificial\\ Instituto Balseiro\\}

\date[]{\lowercase{\today}} 

\maketitle

\section*{Ejercicio 3}
\emph{Proponga un ejemplo para aplicar los conceptos de transferencia de aprendizaje y discuta si los resultados obtenidos eran los esperados.  ¿Cuándo se puede esperar que este tipo de técnicas funciones bien, y cuándo no?}


En este ejercicio se utilizó la red \verb|MobileNet|

\begin{figure}[H]
    \begin{small}
        \begin{center}
            \includegraphics[width=0.5\textwidth]{Figs/ejer3_acc.pdf}
        \end{center}
        \caption{Precisión en función de las épocas}
        \label{fig:ejer3_acc}
    \end{small}
\end{figure}



\begin{figure}[H]
    \begin{small}
        \begin{center}
            \includegraphics[width=0.5\textwidth]{Figs/ejer3_loss.pdf}
        \end{center}
        \caption{Pérdida en función de las épocas}
        \label{fig:ejer3_acc}
    \end{small}
\end{figure}


\section*{Ejercicio 4}
\emph{Utilizando los datos y la arquitectura que considere oportuna describir los distintos procesos para observar lo que la red neuronal ha aprendido.1}

\begin{figure}[H]
    \begin{small}
        \begin{center}
            \includegraphics[width=0.25\textwidth]{Figs/random_init_img_epochs_50.pdf}
        \end{center}
        \caption{}
        \label{fig:random_init}
    \end{small}
\end{figure}


\begin{figure}[H]
    \begin{small}
        \begin{center}
            \includegraphics[width=0.5\textwidth]{Figs/random_filters_epochs_50.pdf}
        \end{center}
        \caption{}
        \label{fig:random_filters}
    \end{small}
\end{figure}

\begin{figure}[H]
    \begin{small}
        \begin{center}
            \includegraphics[width=0.25\textwidth]{Figs/sin_init_img_epochs_50.pdf}
        \end{center}
        \caption{}
        \label{fig:sin_init}
    \end{small}
\end{figure}



\begin{figure}[H]
    \begin{small}
        \begin{center}
            \includegraphics[width=0.5\textwidth]{Figs/sin_filters_epochs_50.pdf}
        \end{center}
        \caption{}
        \label{fig:sin_filters}
    \end{small}
\end{figure}

\begin{figure}[H]
    \begin{small}
        \begin{center}
            \includegraphics[width=0.25\textwidth]{Figs/perr2_init_img_epochs_50.pdf}
        \end{center}
        \caption{}
        \label{fig:perro_init}
    \end{small}
\end{figure}


\begin{figure}[H]
    \begin{small}
        \begin{center}
            \includegraphics[width=0.5\textwidth]{Figs/perr2_filters_epochs_50.pdf}
        \end{center}
        \caption{}
        \label{fig:perro_filters}
    \end{small}
\end{figure}





\end{document}