\input{Preamblev2.sty}

\begin{document}
%%%%%%%%%%%%%%%%%%%%%%%%%%%%%%%%%%Título%%%%%%%%%%%%%%%%%%%%%%%%%%%%%%%%%%%%%%
%%%%%%%%%%%%%%%%%%%%%%%%%%%%%%%%%%%%%%%%%%%%%%%%%%%%%%%%%%%%%%%%%%%%%%%%%%%%%%

\title{Práctica 0: Introducción a Python, Numpy, Matplotlib y Scipy.}
\author{Evelyn~G.~Coronel}

\affiliation{
Aprendizaje Profundo y Redes Neuronales Artificiales\\ Instituto Balseiro\\}

\date[]{\lowercase{\today}} %%lw para lw, [] sin date


\maketitle
%\onecolumngrid


\section*{Ejercicio 1}


\section*{Ejercicio 2}

\begin{figure}[H]
	\centering
	\includegraphics[width=0.5\textwidth]{ejer_2.png}
	\caption{Ejercicio 2}
	\label{fig:ejer2}
\end{figure}
	

\section*{Ejercicio 3}


\section*{Ejercicio 4}

\begin{figure}[H]
	\centering
	\includegraphics[width=0.5\textwidth]{ejer_4.png}
	\caption{Ejercicio 4}
	\label{fig:ejer4}
\end{figure}
	

\section*{Ejercicio 5}


\section*{Ejercicio 6}


\section*{Ejercicio 7}


\section*{Ejercicio 8}


\section*{Ejercicio 9}


\section*{Ejercicio 10}
\begin{figure}[H]
	\centering
	\includegraphics[width=0.5\textwidth]{ejer_10.png}
	\caption{Ejercicio 10}
	\label{fig:ejer10}
\end{figure}
	

\section*{Ejercicio 11}

\begin{figure}[H]
	\centering
	\includegraphics[width=0.5\textwidth]{ejer_11.png}
	\caption{Ejercicio 11}
	\label{fig:ejer11}
\end{figure}
	

\section*{Ejercicio 12}

\begin{figure}[H]
	\centering
	\includegraphics[width=0.5\textwidth]{ejer_12.png}
	\caption{Ejercicio 12}
	\label{fig:ejer12}
\end{figure}
	

\section*{Ejercicio 13: Movimiento del cardumen}

En este ejercicio se vió la utilidad de las clases en Python. Escribí una clase llamada "R2" que intenta describir un vector




\section*{Ejercicio 14}

\begin{figure}[H]
	\centering
	\includegraphics[width=0.5\textwidth]{ejer_14.png}
	\caption{Ejercicio 14}
	\label{fig:ejer14}
\end{figure}
	

\section*{Ejercicio 15}

\begin{figure}[H]
	\centering
	\includegraphics[width=0.5\textwidth]{ejer_15.png}
	\caption{Ejercicio 15}
	\label{fig:ejer15}
\end{figure}
	

\end{document}