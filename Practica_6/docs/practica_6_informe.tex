\input{Preamblev2.sty}
\usepackage{multirow}
\begin{document}

\title{Práctica 6: Método basado en Árboles de Decisión}
\author{Evelyn G. Coronel}

\affiliation{Redes Neuronales y Aprendizaje Profundo para Visión Artificial\\ Instituto Balseiro\\}

\date[]{\lowercase{\today}} 

\maketitle

\section*{Ejercicio 1}

% (a)  Separar los datos en dos particiones, una para datos de entrenamiento/validación (osea, de desarrollo) y otra para test.

% (b)  Entrenar un árbol de decisión para clasificación de la variableHigh. Hacer un plot delárbol (usarscikit-learn) e interpretar los resultados.

% (c)  Entrenar un árbol de decisión para regresión de la variableSales.  Hacer un plot delárbol (usarscikit-learn) e interpretar los resultados.

% (d)  ¿Cuál es el error de test que obtienen en cada caso?  Comparar con el error de entrenamiento y determinar si tienen overfitting o no.

% (e)  Para el árbol de regresión, usarcross-validationpara determinar el nivel óptimo de com-plejidad del árbol.  Busquen cómo usar enscikit-learnla técnica depruningparamejorar la tasa de error de test.

% (f)  Para el caso de regresión,  usar el abordaje tipobaggingpara mejorar el error de test.Comparar con el abordaje de un único árbol de decisión.  Buscar enscikit-learncómo determinar el orden de importancia de los atributos.

% (g)  Usarrandom forestspara mejorar los resultados datos. Comparar el error de test con losabordajes anteriores.  ¿Cambia el orden de la importancia de los atributos?  Hacer unplot con el error de test en función del del hiperparámetromax_featuresque limitael número de atributos a incluir en cada split.  Hacer otro plot equivalente en funcióndemax_depth.1


% (h)  Hacer la misma regresión usandoAdaBoosty comparar errores de test con lo obtenidoconRandom Foresten el punto anterior.2

\end{document}