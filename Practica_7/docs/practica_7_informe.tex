\input{Preamblev2.sty} 
\usepackage{multirow}
\usepackage{paralist}
\begin{document}

\title{Práctica 7: Redes Neuronales Recurrentes }
\author{Evelyn G. Coronel}

\affiliation{Redes Neuronales y Aprendizaje Profundo para Visión Artificial\\ Instituto Balseiro\\}

\date[]{\lowercase{\today}} 

\maketitle

\section*{Ejercicio 1}
\subsection*{Item 1}

Los datos representan la cantidad de pasajeros de aviones por mes desde Enero del 1949 hasta el mes de Diciembre del 1960. Estos datos tienen una oscilación anual y crecimiento en función del tiempo como se ven en la Fig.\,

\subsection*{Item 2}

Para utilizar estos datos para predecir la cantidad de pasajeros en función del tiempo, en el preprocesado se normalizaron los datos de tal forma que varíen entre $0-1$. 

Tomando el set de datos como $x(t)$ donde x representa la cantidad de pasajeros y $t$ representa el mes, se organiza los datos según el desfase $l$ que representa la cantidad de meses y se formatean según el siguiente algoritmo:


	 Si  $l=1$:

	\begin{align*}
		X = x(t) \qquad Y = x(t+1)
	\end{align*}
	
	 Si $l=2$:
	Por ejemplo si queremos obtener $Y$ para el primer  elemento $x(0)$, obtenemos los siguientes vectores:
	\begin{align*}
		X = \big ( x(0), x(1) \big ) \qquad Y = x(2)
	\end{align*}
	Generalizando para cualquier $t$:
	\begin{align*}
		X = \big ( x(t), x(t+1) \big ) \qquad Y = x(t+2)
	\end{align*}
	 Para $l$ arbitrario:
	Generalizando para cualquier $\tilde{t}$:
	\begin{align*}
		X = \big ( x(t), x(t+1),\cdots, x(t+l-1) \big ) \qquad Y = x(t+l)
	\end{align*}

Los datos se formatean de esta forma para hacer que la red aprenda a predecir la cantidad de pasajeros el mes $A$, utilizando la información considerando los $l$ meses anteriores

\subsection*{Items 3, 4 y 5}

Se agrega el ruido gaussiano para probar si la red es capaz de aprender el patrón de la cantidad de pasajeros a pesar del ruido. 

Para entrenar la red recurrente, se separaron los datos de entrenamiento y test en $70\%$ y $30\%$ respectivamente. 

Dado que nuestros conjunto de datos solo contiene la cantidad de pasajeros, por que al redimensionar los datos para la entrada del LSTM, se considera que cantidad de atributos  (\emph{features}) es 1 y el rango de tiempo utilizado para hacer el entrenamiento de $l$ meses.

\subsection*{Item 6 y 7}

La arquitectura de la red utilizada para predecir la cantidad de pasajeros es la siguiente:

\begin{verbatim}
	____________________________________________
	Layer (type)      Output Shape      Param # 
	============================================
	lstm (LSTM)       (None, 1, 100)    40800   
	____________________________________________
	lstm_1 (LSTM)     (None, 1, 100)    80400   
	____________________________________________
	lstm_2 (LSTM)     (None, 1, 80)     57920   
	____________________________________________
	lstm_3 (LSTM)     (None, 80)        51520   
	____________________________________________
	dense (Dense)     (None, 1)         81      
	============================================
	Total params: 230,721
	Trainable params: 230,721
	Non-trainable params: 0
\end{verbatim}

Se utilizó el optimizador Adam con una tasa de aprendizaje de $5\times10^{-5}$ y MSE como función de pérdida.


\subsection*{Item 8}

\subsection*{Item 9}

\subsection*{Item 10}


\end{document}